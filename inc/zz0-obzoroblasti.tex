\section{Анализ предметной области разработки, сравнение с аналогами}

\subsection{Описание предметной области с анализом существующих решений и их недостатков исходя из целей работы и критериев сравнения, которые должны соответствовать искомым решениям поставленных задач.}
Одной из областей в операционных исследованиях является составление и оптимизация расписаний. Когда количество работ, человеческих ресурсов и различных механизмов невелико, то составление расписания вполне посильная для человека задача. Но когда количество объектов, учитываемых при планировании, возрастает, например, до уровня предприятия, то задача перестаёт быть тривиальной даже для целого отдела планирования. В таких случаях становится необходимым использование средств автоматизации планирования, примером которых являются мультиагентные системы или программные комплексы комбинаторной оптимизации - отдельные или в рамках ЕАМ-систем.

В целом можно разделить существующие методы построения и оптимизации подобных моделей на два основных подтипа - Мультиагентные системы (МАС) и Исследование операций (Operations research - далее OR) в рамках целочисленного программирования (программирования в ограничениях).

\subsubsection{МАС}

Мультиагентная система – система, состоящая из множества взаимодействующих друг с другом агентов. Каждый агент в такой системе независим и не имеет представления о системе в целом, но знает свои задачи и потребности, а также задачи и потребности агентов на одном с ним уровне. Для простоты понимания можно условно сказать, что агенты представляют собой нечто схожее с NPC (от англ. Non-Player Character - Неигровой персонаж) в компьютерных играх.

Мультиагентные системы начали широко развиваться в девяностых годах и достигли своего пика к середине двухтысячных годов, после чего парадигма создания агентных сетей как чего-то, требующего отдельных подходов умерла в связи развитием и упрощением программирования в целом.

Согласно статьи Virginia Dignum и Frank Dignum \cite{vfdignum}, несмотря на существование целых организаций по изучению мультиагентных систем, ни за время существования этих организаций, ни за время их работы, МАС не смогли проявить себя ни в одной из заявленных областей, кроме как моделирования поведения персонажей в компьютерных играх \cite{masgames}. Исследователи связывают неудачу МАС со слабой способностью конкурировать с другими компьютерными технологиями того времени, в первую очередь вызванной тем, что из-под пера ассоциации или заинтересованных исследователей так и не вышло платформы МАС, которую могли бы использовать все желающие за пределами «закрытого» сообщества. Второй проблемой при попытке применить исследования агентов на практике, по их мнению, стала неспособность охватить или учесть существенные характеристики проблемы, для решения которой используются агенты. Аналогичные выводы были представлены Городецким и Скобелевым в статье 2017 года \cite{massuck}.

В целом всё-таки можно сказать, что МАС – достаточно перспективное направление, второе дыхание которому могут дать современные разработки в области искусственного интеллекта, нейросетей и программирования.

Главным плюсом (из которого в общих случаях следует и главный минус) мультиагентных систем является скорость реакции на изменения в системе и внешние факторы. Предполагается, что такая система фактически моментально должна реагировать на «происшествия» и сразу же выдавать удовлетворяющее решение с приемлемым качеством. Данное свойство мультиагентных систем делает их максимально пригодными для использования в качестве систем решения задач реального времени или в качестве модели-двойника.

Под моделью-двойником представляется виртуальный клон реальной системы, где все объекты взаимодействия являются агентами. Предполагается, что такие модели будут использоваться, например, для симулирования внештатных ситуаций в системе и оценки реагирования системы и агентов на эти ситуации. 

Так как МАС предлагают в целом «приемлемые решения», то в областях, где важна каждая мелочь и требуется наиболее полный точный результат, такой подход не подойдет. МАС вполне можно использовать на этапах конкретной эксплуатации, где как раз необходимы оперативные принятия решений, а вот в задаче установки оптимальных начальных условий такой подход скорее всего будет неприменим.

\subsubsection{Исследование операций}

Для получения наиболее точных и оптимальных решений применяется дисциплина под названием «Исследование операций». Согласно Википедии \cite{wikicp}, «ИО - дисциплина, занимающаяся разработкой и применением методов нахождения оптимальных решений на основе математического моделирования, статистического моделирования и различных эвристических подходов в различных областях человеческой деятельности. Иногда используется название «математические методы исследования операций».

Методы исследования операций являются основными в решениях таких задач как планирование, транспортные потоки, маршрутизация, упаковка и задачи о назначениях. В варианте, представленном в статье, предлагается использовать методы комбинаторной оптимизации, а именно – целочисленное программирование в ограничениях.

Программирование в ограничениях — это теория решения комбинаторных задач, опирающаяся на большое количество знаний из областей искусственного интеллекта, информатики и исследования операций. В программировании в ограничениях декларативно задаются ограничения для набора переменных входных данных. Ограничения не определяют последовательность действий для получения искомого результата, а указывают свойства и особенности для нахождения конечного результата. 

Оптимизация с ограничениями, или программирование с ограничениями (англ.- constraint programming, далее CP), — это название, данное идентификации выполнимых решений из очень большого набора возможных решений. CP основывается на выполнимости (нахождении выполнимого решения), а не на оптимизации (нахождении оптимального решения) и фокусируется на ограничениях и переменных, а не на целевой функции. Фактически, задача CP может даже не иметь целевой функции - цель может заключаться просто в том, чтобы сузить большое множество возможных решений до более управляемого подмножества путем добавления ограничений к задаче.

\subsection{Разработка возможных направлений проведения исследований и методов решений отдельных задач}

Исходя из требований следует, что разработанная система будет предназначена для перепланирования некоторого множества предварительно заданных в ЕАМ-системе работ.

На текущий момент основным средством ресурсного планирования является фактически ручной метод распределения работ, при котором ответственные люди вручную указывают потребность в работах и ресурсах. При больших объемах работ при таком типе планирования теряется одна из важных потребностей планирования, а именно потребность в актуальности расписания.

Когда количество работ велико, становится физически невозможно одними человеческими силами отследить или как-то исправить пересечения в потребности тех или ресурсов, так как таких коллизий может быть сотни или тысячи даже на промежутке в неделю.

Решение данной проблемы программными средствами тоже сама по себе нетривиальная задача, так как в таком случае на систему необходимо смотреть в целом, а не разрешать отдельные коллизии, а каких-либо общепринятых адекватных решений для данной проблемы до сих пор представлено не было.

Исходя из вышесказанного одним из главных направлений проведения исследований в данной области будет анализ доступных средств и инструментов, подходящих под задачи ресурсного планировании, и вследствие исследование конкретных методов решения поставленных задач уже у выбранного направления.

\subsection{Обоснование применяемых технологий и инструментов.}
Реализация различных методов комбинаторной оптимизации представляется достаточно трудоемкой задачей, поэтому предполагается использование готовых библиотек и фреймворков, реализующих методы исследования операций.

Одними из самых популярных фреймворков для решения задач исследования операций являются открытые GLPK \cite{glpk}, LP\_Solve \cite{lpsolve}, MiniZinc \cite{minizinc}, Google OR-Tools \cite{ortools}, CHOCO \cite{choco} и проприетарные IBM CPLEX \cite{cplex} и Gurobi \cite{gurobi}.

К дальнейшему рассмотрению предлагается три из приведенных выше решений -- Открытый и бесплатный Google OR-Tools, как целевой фреймворк в данной работе, и проприетарные IBM CPLEX и Gurobi, как лидирующие решения в отрасли.

Приведем частичные результаты бенчмарков между Google OR-Tools и IBM CPLEX, полученные в статье "Google vs IBM: A Constraint Solving Challenge on the Job-Shop Scheduling Problem" \cite{orvsplex}, в таблице 1.1.

\begin{table}[H]
	\caption{Сравнение Or-Tools и CPLEX в задаче Job-shop (меньше - лучше)}\label{orplextable}
	\begin{adjustbox}{width=1\textwidth}
		\begin{tabular}{|c|c|c|c|c|}
			\hline \multirow{2}{*}{Номер теста} & \multicolumn{2}{|c|}{Одно ядро} & \multicolumn{2}{|c|}{Четыре ядра} \\
			\cline{2-5} & CPLEX msp (sec) & OR-Tools msp (sec) & CPLEX msp (sec) & OR-Tools msp (sec) \\
			\hline 1 & 1234 (1.9) & 1234 (1.8) & 1234 (3.3) & 1234 (1.6) \\
			\hline 2 & 943 (0.7) & 943 (0.7) & 943 (1.4) & 943 (0.4) \\
			\hline 3 & 656 (1169.3) & 660 & 565 (525) & 661 (1200) \\
			\hline 4 & 682 & 679 & 680 & 679 \\
			\hline 5 & 685 & 695 & 694 & 689 \\
			\hline 6 & 55 (0) & 55 (0) & 55 (0) & 55 (0) \\
			\hline 7 & 930 (3.8) & 930 (5) & 930 (5.9) & 930 (2.9) \\
			\hline 8 & 1165  (1.4) & 1165 (5) & 1165 (0.5) & 1165 (3.4) \\
			\hline 9 & 666 (0) & 666 (0.1) & 666 (0) & 666 (0.1) \\
			\hline 10 & 655 (0.3) & 655 (0.3) & 655 (0.5) & 655 (0.1) \\
			\hline 
		\end{tabular}
	\end{adjustbox}
\end{table}

В тестах производительности между Google OR-Tools и IBM CPLEX стоял вопрос решения проблемы составления расписания работы в цехе (англ. Job-shop scheduling problem), которая приобрела особую известность благодаря своей простой формулировке, приводящей к трудно решаемым задачам.

Наиболее типичным критерием оптимизации является минимизация промежутка времени (англ. makespan), т.е. промежутка времени между началом первой операции и окончанием последней. Проблема представляется в виде набора заданий, которые должны быть обработаны набором машин. Каждое задание - это последовательность операций, каждая операция должна быть обработана определенной машиной и занимает определенное время обработки. Каждое задание имеет определенный порядок операций, который должен соблюдаться. Допустимым решением этой задачи является такая последовательность операций на каждой машине, при которой нет перекрытия времени между двумя операциями на одной машине и соблюдается порядок операций.

Как видим из статьи, по результатам тестов на момент 2017 года фреймворк IBM CPLEX в целом чаще выигрывал у OR-Tools в размере makespan и быстроте получения решения, при этом фреймворк OR-Tools показывал себя лучше в многоядерных тестах большого масштаба.

В другой статье \cite{mipshit} представлены уже результаты сравнения коммерческих IBM CPLEX и Gurobi. По результатам тестов последние версии обоих решателей в целом идут нога в ногу и выдают результаты очень близкие как по качеству, так и по скорости нахождения решений.

По результатам бенчмарков можно сказать, что между фреймворками Google OR-Tools, IBM CPLEX и Gurobi (особенно в многоядерной конфигурации) существует условный паритет по скорости и качеству решения оптимизационных задач, и в целом предугадать, кто окажется впереди в условиях реальной задачи невозможно.

Одним из наиболее доступных фреймворков является Google Or-Tools. OR-Tools - это программное обеспечение с открытым исходным кодом для комбинаторной оптимизации, которая направлена на поиск наилучшего решения проблемы из очень большого набора возможных решений. Последние несколько лет именно OR-Tools среди всех открытых библиотек занимает первые места в соревнованиях MiniZinc Challenge по программированию в ограничениях \cite{zincchal}. Также, в отличие от большинства остальных фреймворков имеет большее количество официально написанных интерфейсов под разные языки программирования – C++ (изначально написан именно на нём), C\#, Java, Python.

Фреймворк Google OR-Tools поддерживает решатели под задачи линейной оптимизации (решения проблемы, смоделированной как набор линейных зависимостей.), целочисленной оптимизации (Смешанное целочисленное программирование - Mixed Integer Programming – MIP; некоторые переменные обязательно представляют собой целые числа) и программирование в ограничениях. Все эти решатели могут быть использованы для решения таких задач, как: решение судоку, диета Стиглера \cite{stigler}, шахматные задачи, планирование, составление оптимальных маршрутов и т.д.

В результате обзора и анализа методов решения поставленной задачи, на данный момент наиболее подходящим методом можно признать метод комбинаторной оптимизации и программирования в ограничениях, так как он явно предоставляет набор устоявшихся практик и решений, показавших себя во множестве проектов.

Основным средством предлагается использовать фреймворк Google OR-Tools, решающий задачи комбинаторной оптимизации, так как данный фреймворк является бесплатным, открытым, может быть использован в коммерческой разработке и покрывает большую часть областей и задач исследования операций.

Мультиагентные системы на данный момент, напротив, по результатам анализа показали себя непригодными к повсеместному использованию, так как не предоставляют доступных средств для реализации и не были широко протестированы в реальных условиях.


\subsection{Уточненная постановка задачи и требования к прототипу решения.}
Имеются следующие начальные условия:
\begin{itemize}
	\item Есть множество запланированных работ (ЗР) на заданном временном интервале. Каждая работа имеет расчетные плановую дату/время начала выполнения и плановую дату/время завершения выполнения (разница между этими датами определяет длительность работы).
	\item Каждая работа имеет перечень исполнителей, необходимых для ее выполнения. Для каждого исполнителя задана потребность, выраженная в чел.*час.  В качестве исполнителя может быть указан один из конкретных исполнителей, имеющихся в штатном расписании.
	\item Имеется подразделение (цех, участок и т.п.), располагающее определенными людскими ресурсами. Состав людских ресурсов определяется штатным расписанием.
\end{itemize}
	Задача:
	
	На основании имеющихся исходных данных постараться построить такой порядок выполнения работ, чтобы в каждый день суммарная потребность в людских ресурсах и механизмах на выполнение всех работ не превышала объем располагаемых ресурсов, заданных в штатном расписании.
	
	Для того, чтобы найти такое решение, система может перемещать исходные ЗР во времени. При этом должны соблюдаться следующие условия:
	\begin{itemize}
		\item длительность работ и потребность в ресурсах, указанные для каждой ЗР, должны сохраняться;
		\item порядок выполнения работ одного типа и для одного и того же объекта должен сохраняться;
		\item может быть задана плановость определенных типов работ.
	\end{itemize}

	Результат:
	
	В результате система должна предложить новые плановые даты выполнения каждой ЗР при условии соблюдения заданных ограничений. Если решения нет, то система должна сообщить об этом, желательно при этом указать, какого ресурса или механизма (по мнению системы) недостаточно для выполнения плана.
	

\subsection{Выводы}

Были проанализированы два подхода к решению проблемы планирования в системах управления активами. Так как в самых популярных на сегодняшний день ЕАМ-системах, таких как 1С ERP, "ТРИМ", зарубежных SAP и Maximo есть лишь стандартные полуавтоматические способы планирования работ, то одним из самых востребованных направлений по улучшению такого функционала является полностью автоматическое планирование по заданным потребностям с учетом актуального распределения ресурсов.

Двумя самыми подходящими для такого функционала подходами являются мультиагентные системы и программирование в ограничениях. По результатам анализа подход с применением мультиагентных систем сложно рассматривать в качестве основного для данного типа задач, так как до сих пор, в том виде в каком он был представлен изначально, не показал реальных практических результатов, а также не было представлено каких-либо общедоступных сред для мультиагентного программирования. Программирование в ограничениях, напротив, предоставляет наиболее понятный набор инструментов и практик, поэтому для решения проблемы ресурсного планирования был выбран именно такой подход.


\clearpage
