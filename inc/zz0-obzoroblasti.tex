\section{Анализ предметной области разработки, сравнение с аналогами}

\subsection{Обзор предметной области}
Пакет 3D-моделирования Blender - свободное и открытое программное обеспечение, предназначенное для создания трехмерной компьютерной графики. В список возможностей пакета входит:
\begin{itemize}
	\item моделирование;
	\item скульптинг;
	\item создание анимации;
	\item симуляция;
	\item рендеринг;
	\item монтаж и обработка видео;
	\item создание рисованной 2D-анимации;
	\item встроенный редактор кода на языке Python, позволяющий взаимодействовать с пакетом Blender.
\end{itemize}

На текущий момент Blender является одним из стандартов в области 3D-моделирования. Пакет широко применяется как в игровых студиях, таких как Ubisoft (серия "Assassin's Creed"), Epic (Unreal Engine, Fortnite), так и в телевизионных сериалах ("Человек в высоком замке" (рис. \ref{mic}), "Кремниевая долина", "Чужестранка") и полнометражных фильмах ("Хардкор", "Парк Юрского периода") \cite{guru-ytb}. Широкое применение получил в науке в области визуализации различных процессов и явлений.

\addimghere{mihc}{1}{Кадр из сериала "Человек в высоком замке" - все объекты, кроме аэропорта были добавлены постобработкой в Blender.}{mic}

\subsubsection{Движки рендера в Blender}

В Blender предустановлено два движка для рендера Cycles и Eevee. Первый предназначен для рендера физически достоверных и сложных сцен, в которых важен реализм, правильное поведение света и отражений, но вычислительно затратен и не поддерживает отображение финального результата в реальном времени.

Движок Eevee, наоборот, специально создавался для версий Blender 2.8x и новее с учетом необходимости поддержки рендера в реальном времени. Хотя и почти все настройки сцены из Cycles подходят для Eevee, их движок просчитывает совершенно по другому, в значительно упрощенной вариации. 

На рисунке \ref{refrac} представлено одно из кардинальных отличий в движках - преломление света. Преломление в Eevee осуществляется путем деформации того, что находится за объектом в соответствии с его нормалями и толщиной. Это означает, что свет не отражается внутри предмета, как в Cycles, и не воспринимает свет, попадающий с разных углов объекта. В Cycles же благодаря полной трассировке пути достигается относительно правильный и физически корректный результат.
 

\addimghere{refractions}{1}{Различия преломления света на примере одного и того же объекта в Cycles и Eevee.}{refrac}

Фактически движок Eevee по своему принципу работы создавался аналогично игровым движкам и спокойно поддерживает рендер в реальном времени, и поэтому именно Eevee был выбран в качестве движка рендера в данной работе.

\subsubsection{Захват движения}

В области захвата движений существуют два основных вида систем. Маркерные системы и безмаркерные.

Маркерная система захвата движений (рис. \ref{mark}). В такой системе несколько камер снимают человека, одетого в специальный костюм с маркерами – датчиками. Человек как бы отыгрывает за виртуального персонажа его роль по сценарию или заданию, а компьютер, получая данные о маркерах с камер – сводит все эти данные в единую 3D-модель, повторяющую движения, мимику и черты актера. Данная система может работать как в реальном времени, так и просто быть основой для дальнейшей обработки.

\addimghere{marker}{0.8}{Маркерная система захвата движений на примере фильма "Железный человек"}{mark}

Безмаркерные системы (рис. \ref{mark}). В данных системах не требуется применение каких-либо специальных устройств, которые бы крепились на человека – безмаркерные системы основаны на применении средств компьютерного зрения и средств распознавания образов. Отсутствие «лишнего» на актере позволяет как ускорить процесс захвата движений, так и записывать более сложные движения без риска травм актёров и повреждения дорогостоящего оборудования.

\addimghere{markerless}{0.8}{Пример работы безмаркерной системы}{markles}

Маркерные системы можно разделить на:
\begin{itemize}
	\item Пассивные оптические системы. В пассивных системах маркеры на актере лишь отражают посланный на них специальными стробоскопами камер инфракрасный свет, показывая тем самым свою позицию на костюме. Минусами данного подхода являются: большие временные затраты на установку и крепление датчиков на актере, плохое различение датчиков при быстром их перемещении или близком расположении относительно друг друга;
	\item В оптических активных системах, напротив, используются системы посылающих сигнал светодиодов и контроллеров, синхронизирующих светодиоды друг с другом, а всю систему с сервером. В остальном работа оптической активной системы схожа с работой оптической пассивной. Из минусов можно выделить невозможность захвата лицевой анимации, необходимость крепления к актеру дополнительного оборудования – контроллера, хрупкость и высокая стоимость;
	\item Магнитные системы. Здесь в качестве маркеров используются магниты, а в качестве «камер» - уловители магнитного потока, которые определяют положения датчиков по изменению ЭМ-поля. Из минусов – подверженность внешнему электрическому воздействию, меньшая зона работы по сравнению с оптическими системами. Остальные минусы схожи с минусами оптических активных систем;
	\item Механические системы используют специальный костюм-скелет, который напрямую отслеживает положение каждого сгиба и вращений суставов. Минусы: сам скелет, различные контроллеры и провода сильно сковывают актера в движениях, нет возможности захвата анимаций лица, возможности определять взаимодействия нескольких актеров в одной сцене;
	\item Гироскопические и инерциальные системы. Здесь данные с сенсоров (например его положение, угол наклона) передаются в компьютер, где уже непосредственно происходит запись этих данных и их обработка. Минусы – нет захвата мимики, высокая стоимость, все равно необходимо наличие оптической или магнитной системы для определения положения актера в сцене.
\end{itemize}

Преимущества систем захвата движений:
\begin{itemize}
	\item результат работы получаем в реальном времени, что значительно снижает затраты относительно покадровой анимации;
	\item объём работы не зависит от сложности или длины задачи, как если бы применялся традиционный покадровый подход. Motion capture позволяет снимать большое количество дублей одной и той же сцены, но в разных стилях и с разной подачей;
	\item сложные движения и реалистичные физические взаимодействия, как например вес или столкновения могут быть легко записаны, при этом будучи физически корректными;
	\item количество анимационных данных, которые могут быть записаны за короткий срок огромны, если сравнивать с традиционными техниками в анимации. Это экономически эффективно, а также позволяет легче укладываться в сроки проекта;
	\item стоимость варьируется от нуля до бесконечночти, что позвляет достаточно легко подобрать тот вариант, который качественно и количественно лучше всех подойдет для проекта;
\end{itemize}

Недостатки:
\begin{itemize}
	\item необходимость в дополнительном ПО, зачастую не входящем в основные пакеты, в которых создаётся анимация;
	\item для некоторых систем захвата движений могут существовать требования к размерам помещения и его электромагнитным свойствам;
	\item при возникновении проблемы в процессе записи анимации становится проще переснять дубль полностью, чем его редактировать, но не все системы позволяют отсматривать результат в реальном времени, чтобы определить, необходимо ли переснять дубль;
	\item трансформации объекта ограничены возможностями объекта захвата, поэтому все дополнительные преобразования всё равно будет необходимо делать позднее;
	\item несовпадение параметров компьютерной модели и объекта съёмки могут привести к различным артефактам.
\end{itemize}



\subsection{Сравнение с аналогами}
Существует несколько распространенных программных продуктов, которые являются аналогами данной работы.

Первый и основной - инструмент Tracking раздела Movie Clip editor, встроенный в пакет Blender \cite{cgmatter-ytb}.

Данный инструмент можно отнести к виду пассивных оптических систем. Для коррректной работы необходимо нанести на лицо специальные маркеры, каждый из которых затем вручную выделяется специальными рамками-трекерами, которые и будут отслеживать изменение координат маркеров на лице и записывать их путь (рис. \ref{marker}).

\addimghere{markercgmatter}{1}{Интерфейс Blender в режиме трекинга маркеров}{marker}

Далее, инструментом Link empty to track точки привязываются к камере уже в 3D пространстве, после чего необходимо вручную привязать соответствующие точки к соответствующим костям (рис. \ref{empty}).

\addimghere{empties}{1}{Модель с привязанными к трекерам костями}{empty}
  
 \clearpage
 
  
Плюсами данного метода являются:
\begin{itemize}
	\item данный инструмент является стандартным для пакета трехмерного моделирования Blender;
	\item много настроек для повышения качества отслеживания маркеров;
	\item может работать как с видеофайлами, так и с последовательностями картинок;
	\item прост и понятен в использовании.
\end{itemize}
Минусы данного метода:
\begin{itemize}
	\item не реального времени;
	\item необходимо использовать маркеры для трекинга;
	\item для лучшего качества отслеживания желательно перевести видео в секвенцию кадров;
	\item связать точки не с моделью лица, с которого делался трекинг можно, но придется пожертвовать "красивым" расположением костей.
\end{itemize}

Второй инструмент - программа FaceCap от niels jansson (рис. \ref{fcap}) \cite{facecap}. 

\addimghere{facecap}{1}{Интерфейс программы FaceCap}{fcap}

Используя TrueDepth сенсор поддерживаемого устройства на iOS, можно записывать и экспортировать FBX файлы, включающие в себя меш, ключи формы (shape keys, blend shapes в различных программах 3D моделирования) и данные об изменениях модели. Также можно экспортировать отдельно данные об анимации в текстовый файл.

Плюсы:
\begin{itemize}
	\item основной функционал бесплатен;
	\item используется безмаркерная оптическая система для захвата движений;
	\item можно в реальном времени следить за захватом лица на базовой модели.
\end{itemize}

Минусы:
\begin{itemize}
	\item только iOS;
	\item полный функционал открывается за 60 долларов;
	\item нет возможности переносить анимацию в реальном времени на модель в Blender.
\end{itemize}

В качестве третьего аналога рассмотрим десктопное проприетарное приложение iClone7 (рис. \ref{ic7}) \cite{iclone}.

\addimghere{iclone7}{1}{Интерфейс программы iClone7}{ic7}

Данное ПО позволяет загрузить готовую модель из Blender со всеми материалами и уже её анимировать в реальном времени с помощью веб-камеры компьютера или ноутбука.

Плюсы iClone7:
\begin{itemize}
	\item захват анимации в реальном времени;
	\item возможность использования готовых моделей из различных пакетов компьютерной графики.
\end{itemize}
Минусы:
\begin{itemize}
	\item стоимость программы iClone7 - 200 долларов;
	\item стоимость аддона 3DXchange 7 для экспортирования заанимированной модели в Blender - 500 долларов.
\end{itemize}


\subsection{Решаемые задачи}


Решаемыми задачами в данной работе являются:
\begin{itemize}
	\item анализ доступных средств пакета Blender и библиотек компьютерного зрения;
	\item проектирование структуры программного модуля;
	\item анализ API Blender3D и функций библиотек ;
	\item написание программного модуля с графическим интерфейсом, работающего внутри пакета Blender.
\end{itemize}


\clearpage
