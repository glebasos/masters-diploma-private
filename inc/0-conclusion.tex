\anonsection{ЗАКЛЮЧЕНИЕ}

В ходе выполнения работы была разработана система, предназначенная для ресурсного планирования в системах управления физическими активами на основе открытых технологий.

Актуальность проведенной работы по разработке системы обосновывается отсутствием подобного функционала в присутствующих на рынке ERP и EAM-системах, таких как 1C-ERP, "TRIM", "Галактика ERP" и так далее, а так же заинтересованностью рынка ЕАМ-систем в подобных сервисах.

Проведен анализ доступных методов для решения вопроса ресурсного планирования:
\begin{itemize}
    \item Мультиагентные системы в целом являются достаточно перспективными в плане использования в ресурсном планировании, так как теоретически способны решить задачу самоорганизации активов предприятия, но данный подход не был широко протестирован в реальных условиях, а сами разработки в этой области, увы, в следствие современных подготов к программированию были фактически заброшены в конце нулевых;
    \item Комбинаторная оптимизация же представляется наиболее подходящим методом решения поставленной задачи на текущий момент. Ресурсное планирование методами комбинаторной оптимизация не представлено в популярных пакетах ЕАМ и ERP-систем, но явно предоставляет набор устоявшихся практик и решений, показавших себя во множестве проектов сходной тематики.
\end{itemize}

Был проведен анализ доступных средств в рамках комбинаторной оптимизации, в котором лучшим образом показал себя фреймворк Google OR-Tools. Данный фреймворк не уступает по качеству и скорости решений как свободным аналогам, так и проприетарным конкурентам, является свободным и открытым, а так же может использоваться и для коммерческой разработки.

Были сформированы требования к системе ресурсного планирования на базе комбинаторной оптимизации.

Были разработаны архитектура системы и программа сервиса планирования работ, а так же прототипа ЕАМ-системы для взаимодействия пользователя с сервисом перепланирования.

Система была создана в рамках экосистемы .Net с использованием фреймворка Blazor Server в качестве основного связующего элемента между сервисом, базой данных и непосредственно прототипом ЕАМ-системы.

Построенная система состоит из открытых и свободных элементов платформы .Net Core, а в качестве СУБД использовался свободный и открытый PostgreSQL, поэтому результирующая система так же является полностью открытой и мультиплатформенной.

После создания прототипа системы ресурсного планирования были проведены экспериментальные исследования, в ходе которых было установлено, что система работоспособна и соответствует поставленной задаче.

Научно-техническая новизна работы состоит в разработке и исследовании применимости модели, основанной на принципах комбинаторной оптимизации для реальных задач в области управления активами и ресурсного планирования.

Практическое применение системы ресурсного планирования заключается в возможности непосредственного её использования в ERP и ЕАМ-системах как некий внутренний инструмент, так и как сторонний подключаемый сервис, работающий с данными из ERP.

За период обучения была опубликована статья в журнале "Инновации. Наука. Образование" на тему "Операционные исследования на основе различных подходов".


\clearpage
