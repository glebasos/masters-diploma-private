\anonsection{ЗАКЛЮЧЕНИЕ}

В ходе выполнения выпускной квалификационной работы бакалавра было разработано программное обеспечение визуализации 3D представления лица с использованием редактора Blender 3D. Решены задачи:
\begin{itemize}
	\item Был разработан безмаркерный программный модуль реального времени визуализации 3D представления лица
	с использованием редактора Blender 3D. Данный модуль позволяет производить перенос и запись мимики лица любого человека в реальном времени на предварительно подготовленную 3D модель, и впоследствии экспортировать модель вместе с анимацией в любое поддерживаемое ПО;
	\item Были решены задачи анализа доступных для программирования средств Blender, анализа и выбора наиболее подходящих средств из библиотек OpenCV и DLib. Была выбрана и предварительно настроена подходящая для работы программного модуля трёхмерная модель;
	\item Был спроектирован и  реализован программный модуль для пакета Blender, включающий в себя интерфейсную часть и часть непосредственной работы алгоритма.
\end{itemize}

По результатам проектирования и реализации был проведен эксперимент для определения работоспособности программного модуля в целом и определения наилучших условий его использования. Модуль может выдавать приемлемые результаты дааже при слабой конфигурации компьютера. Наилучшее расстояние от камеры до пользователя равно 40-75 сантиметрам, что не только обеспечивает максимальную точность и удобство работы с программой, но и соответствует нормам СанПиН по обращению с компьютером. Лучший результат по определению опорных точек модкль выдает при комнатном освещении, но может работать и при тусклом свете.


Разработанное программное обеспечение может быть использовано в качестве альтернативы как встроенных в Blender средств захвата движений, так и внешнего программного обеспечения. В перспективе модуль может быть дополнен средствами для более удобной установки зависимостей и подключения опорных точек к модели.



\clearpage
