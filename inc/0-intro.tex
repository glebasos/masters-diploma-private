\anonsection{Введение}

ЕАМ-системы - чрезвычайно важная часть производственного процесса предприятий. Это инструмент управления производительностью, рисками и затратами, связанными с физическими (производственными) активами организаций.

Для современных производственных процессов актуальными являются задачи
оптимизации: оптимизация маршрутов логистики, рабочего времени, производственных
затрат и т. д. Масштаб таких оптимизационных задач может быть непосилен для решения
сугубо человеческими ресурсами, даже при использовании ЕАМ-систем, поэтому необходимо включать в область решения данных задач программно-аппаратные комплексы, решающие подобные задачи.

С 2011 года мировыми конгломератами ставится вопрос о четвертой промышленной революции - индустрии 4.0. Отходя от её утопических идей, индустрия 4.0 это гигантский шаг в сторону модернизации всех аспектов производства.

Большие данные, интернет вещей, блокчейн, искусственный интеллект, цифровые двойники - все эти технологии, которые уже нашли те или иные способы применения в различных областях науки и техники, только и ждут своего часа применительно к производству и производственным процессам. Уже сейчас самые передовые производства невероятно трудно представить без всех этих технологий.

Так новый передовой завод Intel \cite{linusintel} - полностью автоматизированное цифровое производство, где все самые важные и ответственные роли исполняют именно роботизированные механизмы или автономные дроны с интеллектом роя.

На текущий момент на рынке представлено достаточно много решений, связанных с EAM и ERP системами,  как отечественные:

\begin{itemize}
	\item 1C ERP
	\item "TRIM"
	\item "Галактика ERP"
\end{itemize}
\comment{}{Поменял местами трим и 1с}

так и зарубежные:

\begin{itemize}
	\item SAP ERP
	\item IBM Maximo
\end{itemize}

Все представленные EAM-системы (как отдельные, так и в составе ERP) позволяют в полуавтоматическом режиме планировать работы, то есть указать какое количество ресурсов и когда понадобится для обслуживания тех или иных активов.

Во всех приведённых выше системах данное планирование происходит по потребности, то есть в долгосрочной перспективе не учитывает внезапные изменения, которые могут произойти с ресурсами, и их всевозможные коллизии - ситуации, когда в один момент времени один и тот же ресурс может быть задействован в разных работах.

Исходя из этого можно сказать, что в таких системах стоит до сих пор нерешенная проблема актуализации ресурсного планирования. Именно решение этой проблемы является целью данной работы.

В качестве объекта работы выступают данные, полученные из системы класса EAM, структурированные определенным образом. В общем виде - база данных с набором сущностей вида "работа", "тип работы, "исполнитель" и так далее.

Цель данной работы - создание открытого мультиплатформенного решения, предназначенного для ресурсного планирования в системах управления физическими активами.

В данной работе решается задача ресурсного планирования, то есть распределение заданий по исполнителям в заданном временном интервале с учетом ограниченного количества имеющихся ресурсов.

В результате работы планируется получить готовое программно-техническое решение, которое может взаимодействовать с имеющимися EAM-системами и обеспечивать для них решение задачи ресурсного планирования с заданными граничными условиями.


В процессе работы был проведен анализ доступных средств и методов в области планирования ресурсов, была составлена модель, отражающая в общем виде структуру ресурсов, для которой необходимо проводить процедуры планирования, а также проведен эксперимент, показывающий соответствие модели поставленной задаче.

Работа состоит из Введения, 3 глав и Заключения, содержит список литературных источников из 20 наименований и приложения.

\clearpage
