\section{Проектирование}

Проектирование планировщика ресурсов предполагает собой разработку, состоящую из следующих этапов:
\begin{itemize}
	\item анализ доступных средств платформы .Net;
	\item выбор типа базы данных и её провайдера;
	\item определение алгоритмов, по которым будет работать программный модуль;
	\item написание и отладка программного модуля.
\end{itemize}


\subsection{Проектирование архитектуры прототипа системы решения, включая разработку структуры (компоненты и связи между ними), внутреннего и внешнего функционирования.}

\subsubsection{Анализ доступных средств платформы .Net}

На данный момент платформу .Net можно условно поделить на две части - .Net Framework и .Net Core.

На данный момент платформа .Net Framework является поддерживаемой, но устаревшей, и по рекомендациям Microsoft 111https://docs.microsoft.com/en-us/dotnet/standard/choosing-core-framework-server111 не следует начинать новые проекты на этой платформе. Дополнительно необходимо указать, что платформа .Net Framework не является кроссплатформенной и работает только в операционных системах семейства Windows, чего недостаточно в мире современной разработки на данный момент.

Открытый кроссплатформенный фреймворк .Net Core является преемником .Net Framework. Именно Core версию Microsoft рекомендует для всех новых проектов. Главным отичием новой платформы в первую очередь стала поддержка работы в Unix-системах (Linux, MacOS), соответственно все сервисы, написанные на .Net Core, без каких-либо ограничений могут быть использованы в современной серверной парадигме: Linux серверы, Docker, контейнеризация и виртуализация.

Для данного проекта была выбрана самая акутальная версия фреймворка .Net Core - .NET 6, так как она является LTS-версией, а так же соответствует всем требованиям проекта - открытости и кроссплатформенности.

В качестве платформы для взаимодействия с пользователем был выбран серверный вариант фреймворка Blazor. Blazor - это бесплатный и открытый фреймворк для написания веб приложений на языке C\#, поддерживающий реактивность, то есть работающий напрямую с DOM-элементами, а не HTML страницы. Blazor делится на Blazor WebAssembly - полностью клиентское приложение и на Blazor Server - клиент-серверную реализацию, в которой пользовательская и серверная часть общаются между собой по протоколу SignalR. SignalR - это бесплатная библиотека с открытым исходным кодом для Microsoft ASP.NET, которая позволяет серверному коду отправлять асинхронные уведомления веб-приложениям на стороне клиента. Библиотека включает в себя серверные и клиентские компоненты JavaScript.



\subsubsection{Определение алгоритмов, по которым будет работать программный модуль}

За планирование и/или перепланирование в планировщике будет отвечать библиотека Google OR-Tools. Задача, изложенная выше в первой главе, однозначно подходит под определение задач целочисленного программирования в ограничениях, так как каждый объект виртуальной модели в данной задаче можно определить как целочисленную переменную или их множество, на которые в явном виде можно наложить ряд ограничений и целевых функций, соответствующих условиям задачи.

Для такого типа задач пакет Google OR-Tools предоставляет отдельный класс для решения задач целочисленного программирования в ограничениях - CP-SAT, где CP (от англ. Constraint Programming - Программирование в ограничениях) - непосредственно целочисленное программирование в ограничениях, а SAT (от англ. Satisfiability - Удовлетворимость) - указывает, что целью решения данного класса задач является не нахождение одного конкретного результата, а нахождение одного или нескольких решений, удовлетворяющих ограничениям, наложенным на входные данные.
 
\subsubsection{Выбор типа базы данных и её провайдера.}

В первую очередь при выборе типа баз данных встаёт вопрос об использовании реляционных или нереляционных баз данных. В целом в EAM-системах обычно используются именно реляционные базы данных, так как этому способствует необходимость в четкой иерархической структуре модели представления активов предприятия. Также в EAM-системах редко встречается необходимость в шардинге - то есть в горизонтальном расширении данных для обеспечения масштабируемости. Исходя из этого для данного проекта была выбрана именно реляционная база данных.

В качестве реляционной СУБД был выбран PostgreSQL. PostgreSQL - это бесплатная система управления реляционными базами данных с открытым исходным кодом, в которой особое внимание уделяется расширяемости и соответствию стандарту SQL.

В данный момент ввиду острой необходимости в импортозамещении и независимости от зарубежного ПО, сервисов и средств именно СУБД PostgreSQL видится лучшим вариантом. У российских разработчиков уже есть большой опыт в развитии своих доработок и форков PostgreSQL. Так, например, давно и успешно применяется форк Postgres Pro \textcolor{red}{ССЫЛКА}, СУБД ЛИНТЕР-ВС на базе PostgreSQL используется в ОС МСВС, а в дистрибутив операционной системы Astra Linux Special Edition (версия для автоматизированных систем в защищённом исполнении, обрабатывающих информацию со степенью секретности «совершенно секретно» включительно) включена СУБД PostgreSQL, доработанная по требованиям безопасности информации.

\subsubsection{Проектирование архитектуры}

Желаемая архитектура системы представлена на рисунке ХХ. Предполагается, что разрабатываемая система ресурсного планирования будет являться отдельным сервером планирования, работающим независимо от ЕАМ-системы. Связь между ЕАМ-системой и сервером планирования должна осуществляться посредством наличия общей базы данных и API для передачи команд и промежуточных объектов.

рисунок арч.пнг

Архитектуру самого сервера планирования (рис ХХ) можно представить как API, который посредством контроллера вызывает сервис перепланирования, использующий библиотеку OR-Tools.

рисунок арчсп

Рассмотрим представленную на рисунке ХХХ архитектуру сервера планирования более подробно.

Фронтенд (EAM-система) отправляет команду через API, в котором в привязанном к роуту контроллере выполняются определенные действия - инициализация перепланирования выбранных через фронтенд работ или их же, но уже перепланированных, отправка обратно во фронтенд (ЕАМ-систему) на отображение и подтверждение.

Фактически самой главной частью представленного сервера планирования является сервис OR-Tools. Именно данный сервис отвечает за основную часть - непосредственно процесс планирования предварительно заданных работ. В данном сервисе происходит инициализация модели, задание ограничений для этой модели, а также задается целевая функция, то есть функция, относительно которой будут искаться все удовлетворяющие ограничениям решения для данной модели.


\subsection{Разработка прототипов алгоритмов (как отдельных компонентов и их взаимосвязей, так и в целом системных), структур данных.}

Рассказываю умные слова про требования к модели.

Решение модели эквивалентно нахождению для каждой переменной единственного значения из множества начальных значений (называемого начальной областью), такого, что модель выполнима, или оптимальна, если вы предоставили целевую функцию.

Про ортуловые структуры данных

Вставляю умные закорючки

Все объекты, для которых проводится процесс оптимизации, должны быть представлены в фреймворке Google OR-Tools одним из следующих типов данных:

\begin{itemize}
	\item IntVar; 
	\item BoolVar;
	\item IntervalVar;
	\item OptionalIntervalVar.
\end{itemize}

Тип данных IntVar это объект, который может принимать любое целочисленное значение в определенных диапазонах. В коде объявляется как
\begin{lstlisting}
	var integer_variable = model.NewIntVar(0, 100, "Integer Varible");
\end{lstlisting}
где 0 и 100 это домен переменной, а "Integer Variable" - произвольное имя переменной, задаваемое пользователем.

Тип данных BoolVar представляет собой переменную для хранения булевых значений. Внутри фреймворка BoolVar аналогичен типу IntVar, но с доменом $[0, 1]$. Выделен в отдельный тип данных для удобства программирования.

Пример объявления:
\begin{lstlisting}
	var boolean_variable1 = model.NewBoolVar("Boolean Variable");
\end{lstlisting}
В качестве параметров при объявлении передается только произвольное имя переменной, так как домен по умолчанию лежит в пределах $[0, 1]$

Тип IntervalVar представляет собой интервальную переменную. Интервальная переменная - это и ограничение, и переменная. Она определяется тремя целочисленными переменными: start, size и end.

Она является ограничением, потому что внутри она обеспечивает выполнение того, что start + size == end.

Это также переменная, поскольку она может задаваться в определенных методах ограничения планирования, таких как: NoOverlap, NoOverlap2D, Cumulative

Объявляется так:
\begin{lstlisting}
	var interval_variable = model.NewIntervalVar(start, size, end, "Interval Variable");
\end{lstlisting}
в которой start - предполагаемый момент начала интервала, size - размер (длина) интервала - обычно фиксированное целое число, end - предполагаемый момент окончания интервала. 

В качестве значений start и end могут приниматься либо целые числа - в таком случае, например, суть оптимизации такой переменной заключается в нахождкнии её места на горизонте планирования, либо переменные типа IntVar - тогда оптимизация данной переменной будет зависеть и от также неизвестных переменных определенного домена. 
\clearpage
