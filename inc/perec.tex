\anonsection{Перечень обозначений и сокращений}

ПО -- программное обеспечение.

EAM -- Enterprise Asset Management - система управления активами.

Фреймворк -- программная платформа, определяющая структуру программной системы; программное обеспечение, облегчающее разработку и объединение разных компонентов большого программного проекта.

Makespan -- период времени, который проходит от начала работы до ее завершения.

Job-shop -- оптимизационная проблема в информатике и исследовании операций. Это вариант оптимального планирования заданий. В общей проблеме планирования заданий нам дается $n$ заданий $J_{1}, J_{2}, ..., J_{n}$ с различным временем обработки, которые должны быть запланированы на $m$ машинах с различной вычислительной мощностью, при этом мы пытаемся минимизировать время выполнения (makespan) - общую длину расписания.

SQL -- structured query language - декларативный язык программирования, применяемый для создания, модификации и управления данными в реляционной базе данных, управляемой соответствующей системой управления базами данных.

.NET -- это бесплатная, кроссплатформенная платформа для разработчиков с открытым исходным кодом для создания различных типов приложений.

ORM -- Object-Relational Mapping - объектно-реляционное отображение, технология программирования, которая связывает базы данных с концепциями объектно-ориентированных языков программирования, создавая «виртуальную объектную базу данных».

ADO.NET -- ActiveX Data Object для .NET - технология, предоставляющая доступ к данным и управление ими, хранящимися в базе данных или других источниках.

Entity Framework -- ORM фреймворк с открытым исходным кодом для ADO.NET.

LINQ -- встроенный в .Net язык запросов к данным.
\clearpage
