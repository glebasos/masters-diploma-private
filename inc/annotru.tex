\annotsection{Реферат}

Выпускная квалификационная работа 79 с., 25 рис., 2 табл., 21 источник, 1 прил.

Ключевые слова: операционные исследования, комбинаторная оптимизация, программирование в ограничениях, ресурсное планирование, управление активами.

Актуальность данной работы обусловлена ограниченным функционалом для автоматического ресурсного планирования в ERP и EAM системах.

Цель работы - создание системы ресурсного планирования в системах управления физическими активами на основе открытых технологий.

Объектом исследования является совокупность различных типов ресурсов и запланированных по ним работ на определенном интервале.

Предметом исследования является создание прототипа системы автоматического перепланирования работ в области управления физическими активами.

В процессе работы (исследования) использовались методы математического моделирования в области исследования операций, объектно-ориентированного проектирования и программирования.

Полученные результаты и их новизна:

\begin{itemize}
    \item архитектура программно-аппаратного решения, архитектура программы прототипа системы, модель системы;
    \item разработанный на основе созданных архитектур и алгоритмов прототип системы ресурсного планирования;
\end{itemize}

Основные характеристики:

\begin{itemize}
    \item Основные системные требования к аппаратной конфигурации платформы перепланирования: ОС Windows/Linux/MacOS; процессор с количеством ядер 4 и больше; 16 Гб оперативной памяти и больше;
\end{itemize}

Характеристики прототипа системы:
\begin{itemize}
    \item Ресурсное планирование в системах управления физическими активами;
    \item Возможность взаимодействия пользователя с сервисом планирования посредством прототипа ЕАМ-системы;
    \item Поддержка перепланирования работ разных типов, закрепленных за одним человеком;
\end{itemize}

Поддерживаемые операции: выбор горизонта планирования, запуск сервиса перепланирования, просмотр результатов до и после перепланирования, просмотр статистики по работам, типам работ и рабочим.

Степень внедрения: экспериментальный образец.

Область применения: исследование, внедрение в системы типов ERP и EAM.

Значимость работы заключается в разработке и исследовании применимости модели, основанной на принципах комбинаторной оптимизации для реальных задач в области управления активами и ресурсного планирования.

\clearpage
