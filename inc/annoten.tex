\annotsection{Annotation}

Graduation qualification work 79 pages., 25 pictures., 2 tables., 21 sources, 1 app.

Keywords: operations research, combinatorial optimization, constraint programming, resource planning, asset management.

The relevance of this work is based on the limited functionality for automatic resource planning in ERP and EAM systems.

The purpose of the work is to create a resource planning service for asset management systems based on open-source technologies.


The object of the study is a set of different types of resources and scheduled operations on them at a certain interval.

The subject of the study is the creation of a prototype system for the automatic rescheduling of works in the area of asset management.

In the process of work (research) were used methods of mathematical modeling in the areas of operations research, object-oriented design and programming.

The obtained results and their novelty:

\begin{itemize}
    \item software and hardware solution architecture, prototype system architecture, system model;
    \item a prototype of the resource planning system based on the created architectures and algorithms ;
\end{itemize}

Main characteristics:

\begin{itemize}
    \item Basic system requirements for the hardware configuration of the rescheduling platform: Windows/Linux/MacOS based OS; a processor with 4 or more cores; a processor with 4 or more cores; 16 GB of RAM or more;
\end{itemize}

Application prototype peatures:
\begin{itemize}
    \item Resource planning in EAM systems;
    \item The ability of the user to interact with the planning service through the EAM-system prototype;
    \item The support of the rescheduling of different types of work assigned to a single person;
\end{itemize}

Supported operations: selecting the planning horizon, launching the rescheduling service, viewing the results before and after rescheduling, viewing statistics on jobs, job types and workers.

Degree of implementation: experimental sample.

Scope: research, implementation in systems of ERP and EAM types.

The significance of the work lies in the development and research of the applicability of the model based on the principles of combinatorial optimization in the area of asset management and resource planning.

\clearpage
